% !TeX root = main.tex
\documentclass{./packages/informe}
\usepackage{./packages/caratula}
\usepackage[T1]{fontenc}
\usepackage[spanish]{babel}
\graphicspath{./files/src/.media/}

\begin{document} 

% caratula 
\titulo{TP 1: clasificación de expresiones genómicas} 
\subtitulo{}
\fecha{01 de mayo, 2024}
\materia{Aprendizaje Automático}
% \grupo{Grupo 18}

\integrante{Poner acá integrantes}{LU}{mail}

\maketitle


% palabras clave y resumen
\addtocontents{toc}{\protect\setcounter{tocdepth}{0}}
\section*{resumen}
\addtocontents{toc}{\protect\setcounter{tocdepth}{3}}
Este trabajo práctico busca evaluar la aplicación de diversas técnicas y heurísticas de aprendizaje supervisado 
%---tanto en la evaluación como la construcción de distintos modelos algorítmicos--- 
para la resolución de un problema de investigación. 


En lo que sigue, evaluaremos el uso de los algoritmos de aprendizaje automático \textit{decision trees}, \textit{k-nearest neighbours}, \textit{linear discriminant analysis}, \textit{support vector machines}, \textit{gaussian naïve bayes} y \textit{random forests} para la construcción de un modelo predictivo; realizaremos una búsqueda de hiperparámetros prometedores para mejorar su capacidad, en términos de las métricas \textit{accuracy, auprc} y \textit{aucroc}; e investigaremos los \textit{trade-offs} entre sesgo y varianza en los modelos más prometedores. Con los resultados obtenidos, derivaremos un único clasificador y estimaremos empíricamente su poder de generalización.

% si hay tiempo: 
% Como interesa para el problema de investigación no solamente el desarrollo de técnicas de clasificación, si no también la comprensión de los resultados obtenidos, se dará especial énfasis a la \textit{interpretabilidad} de los modelos obtenidos. 

\vspace{1em}
\noindent Palabras clave: \textit{aprendizaje supervisado}, \textit{evaluación de modelos.}

% \newpage

% contenido
\vspace{1em}
\tableofcontents
\newpage

% descripcion del problema
\section{Descripción del problema}
El desarrollo que se presenta en las próximas secciones se hace en el contexto del siguiente problema de investigación: la detección temprana y diagnóstico preciso de enfermedades como el cáncer\footnote{Dado que este trabajo se centra en la aplicación de técnicas de aprendizaje supervisado, se aclara que tanto el dataset como los resultados obtenidos deben ser considerados \textit{de juguete} a efectos prácticos del problema motivador.}, en base a muestras de células que experimentan distintos tipos de \textit{hiperplasia}\footnote{Se denomina hiperplasia al crecimiento en tamaño de un órgano o tejido, causado por la reproducción anormal y excesiva de sus células. Este tipo de células pueden derivar en tumores.}. 

Para realizarlo, contamos con un dataset $D$ provisto por la cátedra\footnote{El mismo se puede investigar en los archivos adjuntos.} de $n=500$ mediciones de ARN\footnote{Ácido ribonucleico.} que fueron clasificadas según las etiquetas \textit{buen pronóstico} y \textit{mal pronóstico} por expertos del área. Cada medición de ARN es un recorte\footnote{Este recorte se cree crítico en la \textit{transformación} tumoral de una célula, según expertos del área.} de $p=200$ genes, representados por números reales, provenientes de células con algún tipo de hiperplasia en paciente con lesiones pre-tumorales. El proceso de etiquetado es el resultado de un seguimiento periódico de estos mismos pacientes durante un plazo de cinco años.

Cada medición corresponde a un paciente diferente y sus valores oscilan, por lo general, en los rangos $[-a, a]$ para distintos valores de $a \in \mathbb{R_+}$. Los mismos no están estandarizados. 

% separación de datos
\section{Separación de datos}
Lo primero a hacer al construir un modelo de aprendizaje automático es un preprocesamiento de los datos. Esta tarea es crucial para estimar correctamente la performance de nuestro modelo final, y a su vez entrenarlo con datos lo más fieles posibles a su distribución subyacente. 

La idea principal es separar los datos en dos conjuntos. Por un lado, un set de entrenamiento que será utilizado para generar los distintos modelos a lo largo de la investigación. Por el otro, un conjunto de evaluación, usado para medir la performance del modelo final resultante. Esta evaluación del modelo solo se relizará \textbf{una vez} para obtener métricas lo más cercanas a su rendimiento en la vida real. Idealmente, este último conjunto representa un $10-20\%$ de la totalidad de los datos.

De no dividir los datos de esta forma, se evaluaría el rendimiento del modelo resultante con la misma información que se utilizó para entrenarlo. Esto es mala idea ya que, por la misma naturaleza de los modelos, estos se acoplan en cierta medida a los datos con los que fueron entrenados. Luego, al evaluarlos con estas mismas instancias sobreestimaríamos su performance.

Sabiendo que tenemos que realizar estos splits, ¿qué instancias se deberían de incluir en cada uno? Para esto hay que tener en cuenta dos factores:

\begin{itemize}
    \item La proporción de las instancias por clase debe de ser similar en cada split a la del dataset entero.
    \item La división debe de ser realizada al azar para evitar cualquier estructura subyacente en los datos.
\end{itemize}

Con esto en mente, procedimos a separar los datos de la siguiente forma:

\begin{enumerate}
    \item Dividimos el dataset $D$ separando las instancias positivas --\textit{buen pronóstico}-- de las negativas --\textit{mal pronóstico}-- y luego las desordenamos al azar, $D \rightarrow (P, N)$.
    \item Separamos el último $10\%$ de cada lista obteniendo los datos de entrenamiento y evaluación, $P \rightarrow (P_{train}, P_{test})$ y $N \rightarrow (N_{train}, N_{test})$.
    \item Obtenemos los conjuntos finales concatenando las listas del mismo tipo y nuevamente desordenándolas al azar, $(D_{train}, D_{test}) = (P_{train} \cup N_{train},\ P_{test} \cup N_{test})$.
\end{enumerate}

Es importante agregar que, para poder replicar la misma separción azarosa de las instancias, se utilizó una \textit{seed} fija cada vez que se desordenaban los conjuntos.

% construcción de un modelo simple
\section{Construcción de modelos}
\subsection{Un árbol de decisión simple}\label{simple}

Como punto de partida para nuestro análisis, vamos a entrenar un \textit{árbol de decisión} simple sobre los datos de entrenamiento. De esta manera, tendremos un piso respecto a la performance que podemos esperar de cualquier modelo a entrenar. Con \textit{simple} nos referimos a un modelo de poca complejidad, que sea fácilmente interpretable y propenso a subestimar por contar con un sesgo inductivo alto\footnote{En otras palabras, cuyo espacio de hipótesis sea sencillo.}. De manera concreta, el árbol a entrenar tendrá una profundidad máxima de $3$ niveles y usará los hiperparámetros por defecto de la biblioteca \textit{scikit-learn}\footnote{Luego, las hipótesis representables tendrán la forma de conjunciones de a lo sumo tres comparaciones, donde cada comparación evalúa algún umbral para algún atributo particular de las instancias.} para el \textit{DecisionTreeClassifier}.

Para estimar la performance del modelo utilizaremos $5$-fold cross validation estratificado\footnote{Ver la función \textit{cross\_val} en el notebook adjunto. Usamos el splitter \textit{StratifiedKFold} de \textit{scikit-learn}, con \textit{random\_state} $= s$. Para procurar que la comparación entre métricas sea justa, se reseteó el \textit{random\_state} entre corridas.} sobre $D_{train}$, para las métricas de \textit{accuracy}, \textit{aucroc} y \textit{auprc}. Mediremos la performance en entrenamiento y validación por split. También, tomaremos el promedio $\bar\phi$ en ambos casos y, para validación, realizaremos el cálculo del score global\footnote{Notar que todo $D_{train}$ fue parte del set de validación para alguna iteración de cross validation. Luego, podemos calcular $$\Phi = metrica(\bigcup_{i=1}^5\text{Predict}(M^{(i)}, X_{cv_{validation}}^{(i)}), y_{train})$$ para estimar la performance del modelo sin recurrir a promedios.} $\Phi$.

% \vspace{0.5em}
% \begin{enumerate}
%     \item Dividimos el set de entrenamiento en $5$ grupos al azar, procurando que cada grupo mantenga la distribución de clases del set original\footnote{Usamos el splitter \textit{StratifiedKFold} de \textit{scikit-learn} con $k=5$ y \textit{random\_state} $= s$. Para procurar que la comparación entre métricas sea justa, se reseteó el \textit{random\_state} entre corridas.}: $$D_{train} \rightarrow (D_{train}^{(1)}, D_{train}^{(2)}, D_{train}^{(3)}, D_{train}^{(4)}, D_{train}^{(5)})$$
% 
%     \item Por turnos, consideramos un subconjunto como un set de \textit{validación} y al resto como un set de \textit{entrenamiento}.$$
%     cv^{(i)} = (D_{cv_{train}}^{(i)},\ D_{cv_{validation}}^{(i)}) = (D_{train}^{(i)},\ \bigcup_{j \neq i} D_{train}^{(j)})\ \ \forall i: 1 \dots 5$$
% 
%     \item Por cada par $cv^{(i)} $, entrenamos un \textit{árbol de decisión} simple, con los hiperparámetros descriptos: $$M^{(i)} = \text{Train}(Arbol,\ D_{cv_{train}}^{(i)})$$
%     
%     \item Evaluamos la performance de cada modelo $M^{(i)}$ sobre $cv^{(i)}$, utilizando la métrica seleccionada: 
%     \begin{align*}
%         \phi_{train}^{(i)} &= \text{metric}(\text{Predict}(M^{(i)}, X_{cv_{train}}^{(i)}), y_{cv_{train}}^{(i)})\\
%         \phi_{validation}^{(i)} &= \text{metric}(\text{Predict}(M^{(i)}, X_{cv_{validation}}^{(i)}), y_{cv_{validation}}^{(i)})
%     \end{align*}
%     \item Tomamos el promedio de las evaluaciones, para tener un primer estimado del desempeño del modelo, independiente de $D_{train}$. Es decir, un estimación \textit{realista} de la performance:
%     \begin{align*}
%         \bar{\phi}_{train} &= \frac{1}{5}\sum_{i=1}^{5} \phi_{train}^{(i)}\\
%         \bar{\phi}_{validation} &= \frac{1}{5}\sum_{i=1}^{5} \phi_{validation}^{(i)}
%     \end{align*}
%          
%     \item Evaluamos la métrica sobre el conjunto total de datos\footnote{Notar que todo el conjunto fue parte del set de validación para alguna iteración.}, para tener otro estimado del desempeño del modelo, independiente de $D_{train}$:
%     \begin{align*}
%         \Phi_{validation} &= metric(\bigcup_{i=1}^5\text{Predict}(M^{(i)}, X_{cv_{validation}}^{(i)}), y_{train})
%     \end{align*}
% \end{enumerate}
%\vspace{0.5em}

Los resultados se pueden observar en las Figuras \ref{metricas_simple} y \ref{curvas_simple}. Como es esperable, la performance sobre entrenamiento es más alta que la performance sobre validación, para todas las métricas. El hecho que no se logren valores muy altos en entrenamiento da cuenta de un problema de sesgo, como ya habíamos previsto.

\vspace{0.5em}
\begin{figure}[!htbp]
\begin{center}
\begin{tabular}{ |c|c|c|c|c|c|c| } 
\hline
            & accuracy  & accuracy      & auprc     & auprc         & aucroc   & aucroc      \\
            & (train)   & (validación)  & (train)   & (validación)  & (train)   & (validación) \\      
\hline
i=1         & $0.8329$  & $0.7333$      & $0.7676$  & $0.5019$      & $0.7869$  & $0.6270$ \\
i=2         & $0.8189$  & $0.6556$      & $0.7618$  & $0.3545$      & $0.8103$  & $0.6342$ \\
i=3         & $0.8245$  & $0.6333$      & $0.7689$  & $0.5603$      & $0.8460$  & $0.6699$ \\
i=4         & $0.8189$  & $0.7000$      & $0.7551$  & $0.4572$      & $0.7890$  & $0.6221$ \\
i=5         & $0.8083$  & $0.7303$      & $0.7820$  & $0.5590$      & $0.8662$  & $0.6537$ \\
$\bar\phi$  & $0.8207$  & $0.6905$      & $0.7671$  & $0.4869$      & $0.8197$  & $0.6414$ \\
$\Phi$      & -         & $0.6904$      & -         & $0.4810$      & -         & $0.6590$ \\
\hline
\end{tabular}
\end{center}
\caption{Resultados de $5$-fold cross validation estratificado sobre los datos de entrenamiento para las métricas de \textit{accuracy}, \textit{aucroc} y \textit{auprc}, redondeados a cuatro dígitos significativos.}\label{metricas_simple}
\end{figure}

\begin{figure}[!htbp]
    \centering 
    \includegraphics[width=0.45\textwidth]{/files/src/.media/simpleTreeAuprc.png}
    \includegraphics[width=0.45\textwidth]{/files/src/.media/simpleTreeAucroc.png}
    \caption{Curvas \textit{precision-recall}, izquierda, y \textit{receiving operator characteristic}, derecha, para el árbol de decisión \textit{simple}.}
    \label{curvas_simple}
\end{figure}

Tal vez no sorprendentemente, desde el punto de vista de \textit{aucroc}, el modelo funciona apenas un poco mejor que tomar una decisión aleatoria.

Es interesante notar que \textit{accuracy} obtuvo un valor más alto que \textit{auprc} y \textit{aucroc} en validación. Esto puede dar cuenta de la incapacidad de \textit{accuracy} para capturar el efecto de los distintos tipos de errores en un problema con clases desbalanceadas. También, parece interesante que tanto \textit{accuracy} como \textit{auprc} obtuvieron en validación resultados más variados que \textit{aucroc}: \textit{auprc} varió hasta un $\approx 20\%$ entre splits.

Un detalle importante en la interpretación de estos resultados, creemos, es la importancia sobre la decisión de \textit{cuál es la clase positiva}. En particular, \textit{auprc} está sesgado hacia el comportamiento de esta clase, mientras que \textit{aucroc} es más imparcial \cite{Saito}. 


\subsection{Búsqueda en cuadrícula}

Dados los resultados, parece interesante explorar otros hiperparámetros para estimar qué tan capaces pueden llegar a ser los \textit{árboles de decisión} para el problema bajo estudio. Para ello, realizaremos una primer búsqueda en cuadrícula con algunos hiperparámetros provistos por la cátedra. En la próxima sección lo complementaremos con una búsqueda aleatoria más extensiva.

En este caso, evaluaremos las combinaciones de hiperparámetros que titulan la Figura \ref{grid_search}, aplicando $5$-fold cross validation estratificado con métrica \textit{accuracy} promedio\footnote{Por decisión de la cátedra para este ejercicio.}.

\vspace{0.5em}
\begin{figure}[!htbp]
    \begin{center}
        \begin{tabular}{ |c|c|c|c| } 
         \hline
        Altura Máxima   & Criterio de corte & Accuracy (train)  & Accuracy (validación) \\
        \hline
        $3$             & Gini              &  $0.8207$         & $0.6905$  \\ 
        $5$             & Gini              &  $0.9126$         & $0.7127$  \\
        $\infty$        & Gini              &  $1.0000$         & $0.6637$  \\ 
        $3$             & Entropía          &  $0.7890$         & $0.6815$  \\
        $5$             & Entropía          &  $0.8937$         & $0.6503$  \\ 
        $\infty$        & Entropía          &  $1.0000$         & $0.6459$  \\ 
        \hline
        \end{tabular}
    \end{center}
    \caption{\textit{Accuracy} promedio resultante de realizar $5$-fold cross validation estratificado, para \textit{árboles de decisión} con distintos hiperparámetros.} \label{grid_search}
\end{figure}

La figura \ref{grid_search} muestra los resultados obtenidos. Podemos ver una clara tendencia a sobreajustar a medida que aumenta la altura máxima permitida. Por su parte, el criterio Gini obtuvo mejores resultados que Entropía en todas las instancias de evaluación. Algo que llama la atención es que un aumento modesto en la altura máxima permitida (de $3$ a $5$ niveles) causó una mejora en evaluación al usar el criterio Gini, pero una pérdida en el caso de Entropía.


\vspace{1em}
\section{Comparación de algoritmos}
Procedemos a evaluar el uso de distintos algoritmos de aprendizaje supervisado para la resolución del problema. Para cada uno, realizaremos una búsqueda aleatoria con el objetivo de encontrar buenos hiperparámetros, guiados por nuestras propias suposiciones respecto de cuáles pueden llegar a importar y en qué rangos de valores.

La búsqueda se realizará con \textit{RandomizedSearchCV} de \textit{scikit-learn}, utilizando como mecanismo de evaluación $5$-fold cross-validation estratificado con métrica \textit{aucroc} promedio. Se realizarán $100$ iteraciones por algoritmo.

\subsection{decision trees}
Realizamos una última búsqueda de hiperparámetros prometedores para los \textit{árboles de decisión}. La misma se llevó a cabo con el clasificador \textit{DecisionTreeClassifier} de \textit{scikit-learn} para los siguientes hiperparámetros y rangos: 

\begin{itemize}
    \item El criterio de corte: permitimos todos los criterios implementados por el algoritmo. Estos son \textit{Gini}, \textit{Entropy} y \textit{Log loss}. 
    \item La profundidad máxima: permitimos que varie en el rango $[3,\ \lfloor\sqrt{n} \rfloor]$ de manera uniforme, con $n=451$ la cantidad de instancias en $D_{train}$, bajo la suposición que un árbol \textit{corto} es preferible tanto a un \textit{stump} como a un árbol profundo.
    \item La cantidad de atributos máxima a considerar por corte: permitimos que varíe de manera uniforme sobre $[1, p]$, con $p = 200$ la cantidad total de atributos en $D_{train}$.
\end{itemize}

Si bien se exploraron de manera tentativa otros hiperparámetros, se decidió no agregarlos a la búsqueda general, ya que no parecieron influir positivamente en los resultados. A su vez, consideramos que la cantidad de restricciones agregadas es inversamente proporcional a la probabilidad de encontrar una buena combinación de hiperparámetros, en especial cuando la influencia de los hiperparámetros no parece ser igual. 

La Figura \ref{random_tree} muestra los $5$ mejores candidatos obtenidos. Los resultados parecen respaldar la supocisión de que un árbol corto es mejor (notar que $\lfloor\sqrt{n} \rfloor = 21$).

\vspace{0.5em}
\begin{figure}[!htbp]
    \begin{center}
        \begin{tabular}{ |c|c|c|c| } 
         \hline
        Altura Máxima   & Criterio de corte & Atributos Máximos  & aucroc (validación) \\
        \hline
        $3$             & Entropía          &  $135$            & $0.6843$  \\ 
        $3$             & Entropía          &  $14$             & $0.6598$  \\
        $4$             & Log loss          &  $148$            & $0.6517$  \\ 
        $3$             & Log loss          &  $77$             & $0.6513$  \\
        $9$             & Entropía          &  $91$             & $0.6468$  \\ 
        \hline
        \end{tabular}
    \end{center}
    \caption{mejores resultados para la búsqueda aleatoria de hiperparámetros para \textit{árboles de decisión}.} \label{random_tree}
\end{figure}

Si bien el criterio de corte influye, la similitud entre Gini y Entropía llevan a intuir que fue la limitación en la cantidad máxima de atributos a considerar la que llevó a una mejora sustancial con respecto al modelo de la Sección \ref{simple}. Esto puede deberse a que limita la influencia de los atributos más importantes durante la construcción del árbol.

\subsection{k-nearest neighbours}
\subsection{linear discriminant analysis}
La busqueda de hiperparametros se llevo a cabo con el clasificador \textit{LinearDiscriminantAnalysis} de \textit{scikit-learn} usamos los siguientes hiperparámetros: 

\begin{itemize}
    \item Solver: Permitimos los algoritmos basados en $LSQR$, $SVD$ y $EIGEN$.
    \item El metodo de contraccion: Cuando no se este usando Solver=SVD, permitimos que siga una distribucion $Uniforme(0,1)$.
\end{itemize}

Se exploraron otros hiperparámetros, pero optamos por no agregarlos debido a que no parecian influir de forma significativa en los resultados finales. 

\vspace{0.5em}
\begin{figure}[!htbp]
    \begin{center}
        \begin{tabular}{ |c|c|c|c| } 
         \hline
        Solver   & Shrinkage & auc-roc (validación) \\
        \hline
        eigen                   &  0.270007          & $0.8648$  \\ 
        lsqr                    &  0.272309          & $0.8648$  \\
        lsqr                    &  0.266565          & $0.8648$  \\ 
        lsqr                    &  0.274341          & $0.8647$  \\
        eigen                   &  0.288274          & $0.8645$  \\ 
        \hline
        \end{tabular}
    \end{center}
    \caption{Mejores resultados para la búsqueda aleatoria de hiperparámetros para \textit{Linear Discriminant Analysis}.} \label{random_tree}
\end{figure}

La eleccion del Solver Eigen o Lsqr, pareciera no influir pues es mas determinante usar un parametro de contraccion, el uso de svd da como resultado una peor performance. 
\subsection{support vector machines}

\subsection{naïve bayes}
Se examinó por último el rendimiento del clasificador \textit{GaussianNB} de \textit{scikit-learn}. Los hiperparámetros de este algoritmo a optimizar y sus rangos fueron:

\begin{itemize}
    \item Las probabilidades a priori de las clases: permitimos que varíen de manera normal con \textit{media} igual a las probabilidades a priori empíricas (ver Figura \ref{distribucion}) y una \textit{desviación estándar} de $\sigma = 0.1$.
    \item El suavizado de varianza: permitimos que varíe uniformemente en el rango $[0, 1 \times 10^{-2}]$.
\end{itemize}

Como valor predeterminado, \textit{GaussianNB} considera las \textit{probabilidades a priori} de cada clase como sus respectivas proporciones en el dataset. Para ampliar el rango de búsqueda, en lugar de usar las probabilidades originales se utilizaron valores aleatorios cercanos a estas.

Por otro lado, el \textit{suavizado de varianza} indica la cantidad que se suma a todas las varianzas de los atributos para evitar que sean cero\footnote{Su valor representa el porcentaje de la varianza más alta a sumarle a todas las demás.}, lo cual podría producir errores numéricos durante el cálculo de las distribuciones normales. Aunque tiene un valor predeterminado de $1 \times 10^{-9}$, probamos con números al azar hasta $1 \times 10^{-2}$.

Al observar la Figura \ref{naive_bayes}, podemos destacar dos hallazgos interesantes. En primer lugar, parece que el algoritmo tiene baja varianza, ya que al cambiar significativamente las proporciones de las etiquetas su capacidad predictiva se mantiene estable. Por otro lado, da la impresión de que el mejor valor para el suavizado de la varianza de los atributos se encuentra cerca de la cota superior de la búsqueda, es decir, $1 \times 10^{-2}$. Al dejar su valor predeterminado de $1 \times 10^{-9}$, el mejor puntaje obtenido fue de $0.785$, aproximadamente un $7\%$ de reducción respecto a los de la tabla.

\vspace{0.5em}
\begin{figure}[!htbp]
    \begin{center}
        \begin{tabular}{ |c|c|c|c| } 
         \hline
        $P(Y=0)$ & $P(Y=1)$ & Suavizado & aucroc (validación) \\
        \hline
        $0.6320$ & $0.3680$ & $0.975 \times 10^{-2}$ & $0.8408$ \\ 
        $0.6119$ & $0.3881$ & $0.973 \times 10^{-2}$ & $0.8408$  \\
        $0.6579$ & $0.3421$ & $0.982 \times 10^{-2}$ & $0.8406$  \\ 
        $0.8183$ & $0.1817$ & $0.980 \times 10^{-2}$ & $0.8406$  \\
        $0.5157$ & $0.4843$ & $0.960 \times 10^{-2}$ & $0.8406$  \\ 
        \hline
        \end{tabular}
    \end{center}
    \caption{mejores resultados para la búsqueda aleatoria de hiperparámetros para \textit{naïve bayes}.} \label{naive_bayes}
\end{figure}



\vspace{1em}
\section{Diagnóstico sesgo-varianza}
% En este punto, se pide inspeccionar **tres** de sus mejores modelos encontrados hasta ahora de cada familia de modelos: la mejor configuración para el árbol de decisión, la mejor configuración para LDA y la mejor configuración para SVM. Para ello:

% 1. Graficar curvas de complejidad para cada modelo (excepto para LDA), variando la profundidad en el caso de árboles, y el hiperparámetro C en el caso de SVM. Diagnosticar cómo afectan al sesgo y a la varianza esos dos hiperparámetros.
% 2. Graficar curvas de aprendizaje para cada modelo. En base a estas curvas, sacar conclusiones sobre si los algoritmos parecen haber alcanzado su límite, o bien si aumentar la cantidad de datos debería ayudar.
% 3. Construir un modelo **RandomForest** con 200 árboles. Explorar para qué sirve el hiperparámetro max_features y cómo afecta a la performance del algoritmo mediante una curva de complejidad. Explicar por qué creen que se dieron los resultados obtenidos. Por último, graficar una curva de aprendizaje sobre los parámetros elegidos para determinar si sería útil o no conseguir más datos.

% **Atención**: Tener en cuenta que debemos seguir utilizando AUC ROC como métrica para estas curvas.

% Para cada método pueden incluir hasta una carilla de texto y los gráficos que considere relevantes.

Procedemos en esta sección a inspeccionar los mejores modelos encontrados para los clasificadores \textit{árboles de decisión}, \textit{LDA} y \textit{SVM}. En particular, buscamos entender su comportamiento al aumentar su complejidad (variando alguno de sus hiperparámentros), trazar sus curvas de aprendizaje, y sacar conclusiones sobre sus varianzas y sesgo. Por último, los compararemos con un clasificador de tipo \textit{random forest}.

\subsection{decision trees}
Podemos observar de la Figura \ref{random_tree} que los mejores hiperparámetros para este clasificador son:

\begin{itemize}
    \item \textit{Atributos Máximos} = 135.
    \item \textit{Criterio de Corte} = Entropía.
    \item \textit{Altura Máxima} = 3
\end{itemize}

Utilizamos este último para graficar las curvas de complejidad observadas en la Figura \ref{decisionTreeComplexity}. En ellas podemos apreciar que, respaldando los resultados de este clasificador en la sección anterior, obtenemos la mejor performance utilizando \textit{altura máxima} 3 y 4. Podemos intuir que durante la búsqueda aleatoria de hiperparámentros la combinación con \textit{altura máxima} 4 no se habría probado, de otro modo hubiese aparecido como la mejor.  

A partir de los clasificadores de altura mayor a 7 se puede ver cómo el modelo se  acopla completamente a los datos de entrenamiento, generando un \textit{overfitting}. A su vez, sus respectivas performances en los datos de validación decrementan desde el mejor valor de \textit{altura máxima} = 4 hasta estabilizarse en las alturas mayores a 8. Esto podría indicar que el modelo ya separó todas las instancias en sus respectivas hojas, y dejó de cambiar sus reglas de corte. 

\begin{figure}[!htbp]
    \centering
    \includegraphics[width=0.75\textwidth]{/files/src/.media/decisionTreeComplexity.png}
    \caption{Curvas de complejidada para el clasificador \textit{decision tree}, mostrando la variación del AUC ROC en datos de entrenamiento y validación a medida que se aumenta su profundida máxima. Las lineas verticales denotan la varianza para cada valor del atributo.}
    \label{decisionTreeComplexity}
\end{figure}

Con respecto a la varianza, podemos observar que el rendimiento de los modelos varía considerablemente según el dataset utilizado en entrenamiento. Esto puede apreciarse en las barras de error\footnote{Estas provienen de las mediciones de AUC ROC para cada fold de validación.}, donde en algunos casos como para \textit{altura máxima} = 2 y 5 hubo una diferencia de performance en más de un $10\%$. A su vez, se puede ver cómo se van separando ambas curvas a medida que aumenta la profundidad máxima, lo cual nuevamente muestra que el rendimiento del modelo está fuertemente afectado por los datos utilizados para entrenarse. Por último, que el modelo realice \textit{overfitting} para valores más altos del hiperparámetro es un indicador que la varianza aumenta junto a él. 

Por otro lado, los resultados del algoritmo en validación se encuentran siempre con un AUC ROC por debajo de $0.75$, y con media aproximadamente en $0.6$, con lo cual parecería tener poca capacidad predictiva. En ese sentido, sin importar el valor del hiperparámentro parecería ser un algoritmo incapaz de acercarse a la distribución subyacente de los datos, produciendo \textit{underfitting}, presentando un sesgo alto.



\subsection{SVM}
De la Figura  que los mejores hiperparámetros para este clasificador son:

\begin{itemize}
    \item \textit{Atributos Máximos} = 135.
    \item \textit{Criterio de Corte} = Entropía.
    \item \textit{Altura Máxima} = 3
\end{itemize}

Utilizamos este último para graficar las curvas de complejidad observadas en la Figura \ref{decisionTreeComplexity}. En ellas podemos apreciar que, respaldando los resultados de este clasificador en la sección anterior, obtenemos la mejor performance utilizando \textit{altura máxima} 3 y 4. Podemos intuir que durante la búsqueda aleatoria de hiperparámentros la combinación con \textit{altura máxima} 4 no se habría probado, de otro modo hubiese aparecido como la mejor.  

\begin{figure}[!htbp]
    \centering
    \includegraphics[width=0.75\textwidth]{/files/src/.media/decisionTreeComplexity.png}
    \caption{Curvas de complejidada para el clasificador \textit{decision tree}, mostrando la variación del AUC ROC en datos de entrenamiento y validación a medida que se aumenta su profundida máxima.}
    \label{SVMComplexity}
\end{figure}

A partir de los clasificadores de altura mayor a 7 se puede ver cómo el modelo se  acopla completamente a los datos de entrenamiento, generando un \textit{overfitting}. A su vez, sus respectivas performances en los datos de validación decrementan desde el mejor valor de \textit{altura máxima} = 4 hasta estabilizarse en las alturas mayores a 8. Esto podría indicar que el modelo ya separó todas las instancias en sus respectivas hojas, y dejó de cambiar sus reglas de corte. 

\subsection{random forest}


\vspace{1em}
\section{Evaluación de performance}
De los resultados obtenidos a lo largo de este informe, vemos que los modelos más prometedores parecen ser aquellos basados en los algoritmos de \textit{linear discriminant analysis} y \textit{support vector machines}. En la búsqueda aleatoria ambos obtuvieron buenos puntajes. El anterior obtuvo un \textit{aucroc} promedio de $0.8649$ y el posterior de $0.8960$.

A pesar de la ventaja aparente del segundo, en el diagnóstico de sesgo y varianza continuamos por observar el comportamiento de ambos y notamos que el modelo basado en el algoritmo de \textit{support vector machines} parece tener un sesgo mayor y una tendencia a sobreajustarse a los datos, incluso al contar con una cantidad de datos grande de entrenamiento.

Luego, consideramos que el mejor modelo ---tanto en el sentido del \textit{aucroc} promedio obtenido, como en el sentido de la confianza que nuestra experimentación nos permite depositar en esta métrica--- es aquel obtenido por el algoritmo de \textit{linear discriminant analysis} con los hiperparámetros encontrados.  

\vspace{1em}
\section{Conclusiones}
A lo largo de este informe, experimentamos con distintos algoritmos de aprendizaje automático y evaluamos sus respectivos rendimientos. En base a los datos obtenidos, se eligió el clasificador que para nosotros mejor aproxima la solución al problema y estudiamos su performance final. 

Creemos interesante aprovechar esta sección para mencionar algunos problemas con los que nos encontramos y proponer mejoras metodológicas para futuras investigaciones.

Uno de los aspectos sobre los cuales no profundizamos fue el hecho de normalizar o no los datos. Para algunos algoritmos, tales como \textit{SVM} o \textit{KNN}, se sabe que hacerlo puede tener una implicancia importante sobre los resultados. Durante las etapas de entrenamiento y validación exploramos esta dimensión, pero optamos por no incluirla al no mejorar los resultados. Esto igualmente podría ser un error de nuestra parte, ya que a pesar de arrojar peores resultados tal vez conduce a modelos más robustos.

A su vez, nunca se le dió un enfoque a realizar un análisis previo del dataset recibido, tal como apreciar la distribución de los atributos (por ejemplo a través de histogramas). Nosotros realizamos en forma exploratoria un pequeño análisis respecto a estas distribuciones observando un resumen estadístico, donde notamos una media por lo general rondando el $0$ y algunos indicios de que las mismas se comportarían de manera normal. Consideramos que en futuras investigaciones deberíamos de realizar una exploración mayor respecto a los atributos con los que contamos, por ejemplo a través de un análisis de sus importancias.

Por otro lado, no se contempló si la métrica \textit{auc roc} promedio es la más indicada para el dominio de este problema. Por ejemplo, \cite{Saito} recomienda usar la métrica \textit{auprc} cuando la proporción de las clases es desbalanceada, como es en este caso. Esto está también relacionado a cuál es la clase positiva. Si bien no se investigó si convendría utilizar el \textit{mal pronóstico}, al menos sabemos que la métrica \textit{auc roc} es más imparcial respecto a cuál de las clases representa la etiqueta. Estas elecciones son muy importantes y futuras investigaciones deberían abordarlas a mayor consciencia.

Por último, vale la pena mencionar las limitaciones de la manera en la que realizamos la búsqueda de los hiperparámetros. Consideramos que la cantidad de iteraciones realizada fue baja respecto a los espacios de búsqueda que decidimos explorar. A su vez, es muy posible que existan mejores métodos de sampleo que aquellos que empleamos. Otro aspecto que no fue explorado fue la posibilidad de realizar una búsqueda en cuadrícula en vez de aleatoria, sobre todo para los hiperparámetros cuyos valores son discretos. Creemos que una mejor investigación pondría mayor énfasis en los métodos a través de los cuales realizar este tipo de búsquedas.

% apendice
% \section{Apéndice}
% \input{files/src/apendice.tex}
% \newpage

%bibliografia - requiere que haya citas
\newpage
\bibliographystyle{plain}
\bibliography{./files/citations.bib}

\end{document}
