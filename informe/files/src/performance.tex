De los resultados obtenidos a lo largo de este informe, vemos que los modelos más prometedores parecen ser aquellos basados en los algoritmos de \textit{linear discriminant analysis} y \textit{support vector machines}. En la búsqueda aleatoria ambos obtuvieron buenos puntajes. El anterior obtuvo un \textit{aucroc} promedio de $0.8649$ y el posterior de $0.8960$.

A pesar de la ventaja aparente del segundo, en el diagnóstico de sesgo y varianza notamos que el modelo basado en el algoritmo de \textit{support vector machines} parece tener un sesgo mayor y una tendencia a sobreajustarse a los datos, incluso al contar con una cantidad de datos grande de entrenamiento.

Luego, consideramos que el mejor modelo ---tanto en el sentido del \textit{aucroc} promedio, como en el sentido de la confianza que nuestra experimentación nos permite depositar en esta métrica--- es aquel obtenido por el algoritmo de \textit{linear discriminant analysis} con los hiperparámetros encontrados.  

Procedimos, finalmente, a entrenar un único modelo a través de este algoritmo con todos los datos de entrenamiento. Las Figuras \ref{metricas_final} y \ref{curvas_final} muestran su performance en evaluación. Estos valores conforman nuestra estimación empírica del poder de generalización del modelo.

Importa mencionar que hubo una pérdida del $6\%$ con respecto al mejor resultado obtenido durante validación\footnote{En la evaluación de las curvas de complejidad y aprendizaje.}. Sin embargo, esto es esperable, dado que la validación cruzada fue parte del proceso de entrenamiento y, por ende, da una mirada optimista de la performance real del modelo.

\vspace{0.5em}
\begin{figure}[!htbp]
    \begin{center}
        \begin{tabular}{ |c|c| } 
         \hline
        Métrica         & Valor \\
        \hline
        accuracy (test) &  0.8235 \\
        auprc (test)    &  0.7647 \\
        aucroc (test)   &  0.8321 \\
        \hline
        \end{tabular}
    \end{center}
    \caption{\textit{Accuracy}, \textit{aucroc} y \textit{auprc} obtenidos para el modelo final, evaluados sobre $D_{test}$.} \label{metricas_final}
\end{figure}

\begin{figure}[!htbp]
    \centering 
    \includegraphics[width=0.45\textwidth]{/files/src/.media/finalAuprc.png}
    \includegraphics[width=0.45\textwidth]{/files/src/.media/finalAucroc.png}
    \caption{Curvas \textit{precision-recall}, izquierda, y \textit{receiving operator characteristic}, derecha, para el modelo final.}
    \label{curvas_final}
\end{figure}
