El desarrollo que se presenta en las próximas secciones se hace en el contexto del siguiente problema de investigación: la detección temprana y diagnóstico preciso de enfermedades como el cáncer\footnote{Dado que este trabajo se centra en la aplicación de técnicas de aprendizaje supervisado, se aclara que tanto el dataset como los resultados obtenidos deben ser considerados \textit{de juguete} a efectos prácticos del problema motivador.}, en base a muestras de células que experimentan distintos tipos de \textit{hiperplasia}\footnote{Se denomina hiperplasia al crecimiento en tamaño de un órgano o tejido, causado por la reproducción anormal y excesiva de sus células. Este tipo de células pueden derivar en tumores.}. 

Para realizarlo, contamos con un dataset $D$ provisto por la cátedra\footnote{El mismo se puede investigar en los archivos adjuntos.} de $n=500$ mediciones de ARN\footnote{Ácido ribonucleico.} que fueron clasificadas según las etiquetas \textit{buen pronóstico} y \textit{mal pronóstico} por expertos en el área. Cada medición de ARN es un recorte\footnote{Este recorte se cree crítico en la \textit{transformación} tumoral de una célula, según expertos en el área.} de $p=200$ genes, representados por números reales, provenientes de células con algún tipo de hiperplasia en paciente con lesiones pre-tumorales. El proceso de etiquetado es el resultado de un seguimiento periódico de estos mismos pacientes durante un plazo de cinco años.

Cada medición corresponde a un paciente diferente y los rangos de sus atributos no están estandarizados. 