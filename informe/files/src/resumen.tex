Este trabajo práctico busca evaluar la aplicación de diversas técnicas y heurísticas de aprendizaje supervisado ---tanto en la evaluación como la construcción de distintos modelos algorítmicos--- en el marco de un problema de investigación específico: la detección temprana y diagnóstico preciso de enfermedades como el cáncer, en base a muestras de células que experimentan distintos tipos de \textit{hiperplasia}\footnote{Se denomina hiperplasia al crecimiento en tamaño de un órgano o tejido, causado por la reproducción anormal y excesiva de sus células. Este tipo de células pueden derivar en tumores.}. 

Para ello, contamos con un dataset $D$ provisto por la cátedra\footnote{Dado que este trabajo se centra en la aplicación de técnicas de aprendizaje supervisado, se aclara que tanto el dataset como los resultados obtenidos deben ser considerados \textit{de juguete} a efectos prácticos del problema motivador.} de $n:=500$ mediciones de ARN que fueron clasificadas según las etiquetas \textit{buen pronóstico} $:=1$ ó \textit{mal pronóstico} $:=0$ por expertos del área. Cada medición de ARN es un recorte\footnote{Este recorte se cree crítico en la \textit{transformación} tumoral de una célula, según expertos del área.} de $p:=200$ genes, representados por números reales, provenientes de células con algún tipo de hiperplasia en paciente con lesiones pre-tumorales\footnote{Importa aclarar que cada medición corresponde a un paciente distinto.}. El proceso de etiquetado, por su parte, es el resultado de un seguimiento periódico de estos mismos pacientes durante un plazo de cinco años.

En lo que sigue, evaluaremos ---en términos de \textit{accuracy, auprc} y \textit{aucroc}--- la aplicación de los algoritmos de aprendizaje automático \textit{decision trees}, \textit{k-nearest neighbours}, \textit{linear discriminant analysis}, \textit{support vector machines} y \textit{gaussian naïve bayes}; realizaremos una búsqueda de hiperparámetros prometedores para mejorar su poder predictivo; e investigaremos los \textit{trade-offs} entre sesgo y varianza en la estimación de la capacidad de los modelos más prometedores. Con los resultados obtenidos, derivaremos un único modelo para el cual estimaremos empíricamente su poder de generalización.

% si hay tiempo: 
% Como interesa para el problema de investigación no solamente el desarrollo de técnicas de clasificación, si no también la comprensión de los resultados obtenidos, se dará especial énfasis a la \textit{interpretabilidad} de los modelos obtenidos. 

$\\$
Palabras clave: \textit{aprendizaje supervisado}, \textit{evaluación de modelos.}