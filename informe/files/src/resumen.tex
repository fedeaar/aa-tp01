Este trabajo práctico busca evaluar la aplicación de diversas técnicas y heurísticas de aprendizaje supervisado 
%---tanto en la evaluación como la construcción de distintos modelos algorítmicos--- 
para la resolución de un problema de investigación. 


En lo que sigue, evaluaremos el uso de los algoritmos de aprendizaje automático \textit{decision trees}, \textit{k-nearest neighbours}, \textit{linear discriminant analysis}, \textit{support vector machines}, \textit{gaussian naïve bayes} y \textit{random forests} para la construcción de un modelo predictivo; realizaremos una búsqueda de hiperparámetros prometedores para mejorar su capacidad, en términos de las métricas \textit{accuracy, auprc} y \textit{aucroc}; e investigaremos los \textit{trade-offs} entre sesgo y varianza en los modelos más prometedores. Con los resultados obtenidos, derivaremos un único clasificador y estimaremos empíricamente su poder de generalización.

% si hay tiempo: 
% Como interesa para el problema de investigación no solamente el desarrollo de técnicas de clasificación, si no también la comprensión de los resultados obtenidos, se dará especial énfasis a la \textit{interpretabilidad} de los modelos obtenidos. 

\vspace{1em}
\noindent Palabras clave: \textit{aprendizaje supervisado}, \textit{evaluación de modelos.}
