Se denomina \textbf{hiperplasia} al crecimiento en tamaño de un órgano o tejido, causado por la reproducción anormal y excesiva de sus células. Son estas mismas células las que, por alguna razón que deseamos comprender, derivan potencialmente en tumores. Es de nuestro interés descifrar este comportamiento, y de ese modo predecir si un paciente que experimentó hiperplasia terminará en una recaida o mejorará. 

Con esto en mente, se recopilaron datos de 500 pacientes que presentaban lesiones pre tumorales y antecedentes familiares. Por cada uno, se extrajeron muestras genéticas de las cuales contamos con recortes, enfocados en la expresión de 200 genes en particular que se creen críticos en la transformación tumoral. Cada una de estas muestras fue etiquetada como \textit{buen} o \textit{mal pronóstico} en base a si ese mismo paciente mostró indicios de nuevas hiperplasias o similares.

A partir de estos datos, el objetivo de este informe es obtener un buen modelo de inteligencia artificial que determine las etiquetas de pacientes nuevos. A su vez, determinaremos qué significa que este modelo sea exitoso y sobre qué métricas lo mediremos.